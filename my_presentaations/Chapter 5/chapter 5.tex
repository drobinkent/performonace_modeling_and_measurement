%%%%%%%%%%%%%%%%%%%%%%%%%%%%%%%%%%%%%%%%%%%%%%%%%%%%%%%%%%%%%%%%%%%%%%%
%
%   Presentation of Beamer UNL Theme
%   Beamer Presentation by Chris Bourke
%
%%%%%%%%%%%%%%%%%%%%%%%%%%%%%%%%%%%%%%%%%%%%%%%%%%%%%%%%%%%%%%%%%%%%%%%

\documentclass{beamer}

\usetheme[hideothersubsections]{UNLTheme}
\usepackage[postscript]{ucs}
\usepackage[utf8x]{inputenc}

\title{Performance Modeling and
Design of Computer Systems- Ch 5 \\
Sample Paths, Convergence,
and Averages
}
\author{Debobroto Das Robin} %
\institute{Kent State University}
\date{Spring 2020}




\begin{document}

%{% open a Local TeX Group
%\setbeamertemplate{sidebar}{}
\begin{frame}
        \titlepage
        \begin{center}
    \href{mailto:drobin@kent.edu}{\color{blue}{\texttt{drobin@kent.edu}}}
        \end{center}
\end{frame}

\begin{frame}
\frametitle{Overview} % Table of contents slide, comment this block out to remove it
\tableofcontents % Throughout your presentation, if you choose to use \section{} and \subsection{} commands, these will automatically be printed on this slide as an overview of your presentation
\end{frame}



\section{Introduction}


\begin{frame}
\frametitle{Convergence}
\framesubtitle{\textbf{\textit{}}}
\begin{itemize}
\item \textbf{Definitiion}  A sequence $\{a_n : n = 1, 2, \dot{...}\}$ converges to b as $ n \rightarrow \infty$ over a sample path , mathmetically
$$\lim_{n\to\infty} a_n = b$$
\item Convergence for random variable ( 2 erquivalent def)
	\begin{itemize}
	\item The sequence of random variables $\{Y_n : n = 1, 2, \dot{...}\}$ 			converges almost surely to $\mu$, written
			$Y_n \rightarrow \mu , as :\ n \rightarrow \infty$
	\item  The sequence converges with probability 1, written
			$Y_n \rightarrow ,\mu :\ as :\ n \rightarrow \infty$ with 							probability $p =$  1
		if, $\forall k > 0, P \{ \lim_{n \rightarrow \infty} ( Y_n - \mu)> k\}= 0$

	\end{itemize}
\end{itemize}
	
\end{frame}



\begin{frame}
\frametitle{Convergence}
\framesubtitle{\textbf{\textit{}}}
\begin{itemize}
\item \textbf{Sample path} of a stochastic process is a particular realisation of the process, i.e. a particular set of values $Y(t)$ for all t

Example: A coin is tossed n times. the probability of head in each toss forms a path. This is a sample path  $\frac{1}{2}$,$0$, $\cdots{},\frac{1}{2}$

\item Convergence is considered over a sample path
	
\end{itemize}	    
    
\end{frame}

\section{Strong and Weak Laws of Large Numbers}
\begin{frame}
\frametitle{Weak Law of large numbers}
\framesubtitle{\textbf{\textit{}}}
\begin{itemize}
\item Let $X_1 , X_2 , X_3 , . . .$ be i.i.d. (independent and identically distributed) random variables with mean $E [X]$ . Let
$$S_n = {\sum}_{i=1}^{n} X_i$$ and $$Y_n = \frac{S_n}{n}$$

Then \textbf{$Y_n$ converges in probability to $E[X]$}, 
mathemetically $Y_n \rightarrow E[X]$ , as $n \rightarrow \infty$

Equivalently, $\forall k > 0, \lim_{n \rightarrow \infty} P \{  | Y_n - E[X]| > k)\}= 0$


\item
	
\end{itemize}	    
    
\end{frame}

\begin{frame}
\frametitle{Strong Law of large numbers}
\framesubtitle{\textbf{\textit{}}}
\begin{itemize}
\item Let $X_1 , X_2 , X_3 , . . .$ be i.i.d. (independent and identically distributed) random variables with mean $E [X]$ . Let
$$S_n = {\sum}_{i=1}^{n} X_i$$ and $$Y_n = \frac{S_n}{n}$$

Then \textbf{$Y_n$ converges almost \textit{surely} to $E[X]$}, 
mathemetically $Y_n \rightarrow E[X]$ , as $n \rightarrow \infty$

Equivalently, $\forall k > 0, \lim_{n \rightarrow \infty} P \{  | Y_n - E[X]| \geq k)\}= 0$


\item
	
\end{itemize}	    
    
\end{frame}


\begin{frame}
\frametitle{Example of Strong \& Weak Law of large numbers}
\framesubtitle{\textbf{\textit{}}}
\begin{itemize}
\item Let $X_i$ are all 0/1 random variables (coin toss) with mean $\frac{1}{2}$
\item Each sample path is an infinite sequence of coin flips

\item \textbf{Strong Law} : for \textbf{“almost every”} sample path, if we average the coin flips far out enough (large $n$) along the path, we will get convergence to $\frac{1}{2}$

\item \textbf{Weak Law} : Says convergence is not almost sure. Convergence will happen with a probability. 
	
\end{itemize}	    
    
\end{frame}

\section{Concept of Average}

\begin{frame}
\frametitle{Time Average versus Ensemble Average}
\framesubtitle{\textbf{\textit{}}}
\begin{itemize}
\item Scenario: a single FCFS queue,  at every second a new job is
added to the queue with probability $p$ and at every second the job in service (if there is) completed with probability $q$ , where $q > p$. Let $N (v) =$ number of jobs in the system at time $v$ . 

\item What is the avg. number  of jobs in the queue

\item  \textbf{Time Average}: $$\overline{N}^{Time Avg} = \lim_{t \rightarrow \infty} \: \frac{\int_0^t \: N(v)dv}{t}$$

\item  \textbf{Ensemble Average}: $$\overline{N}^{Ensemble} = \lim_{t \rightarrow \infty} \: E[N(t)] = {\sum^{\infty}_{i=0}} ip_i$$

Where, 
$$p_i = \lim_{t \rightarrow \infty} \: P\{ N(t) = i \}$$

	
\end{itemize}	    
    
\end{frame}



\begin{frame}
\frametitle{Time Average versus Ensemble Average}
\framesubtitle{\textbf{\textit{Interpretation: Time Average}}}
\begin{itemize}
\item Consider the FCFS scenario again 


\item  \textbf{Time Average}: Consider a sample path over a long period $t$, mointor \# of jobs in system , then take avg

Example: queue  start empty. 
at time 1, an arrival , no departure N (1) = 1. 

at time 2, an arrival , no departure N (2) = 2). 

At time 3, an arrival , no departure N (3) = 3 

at time 4, no arrival and a departure N (4) = 2, etc. 

The average number of jobs in the system by time 4 for this process is (0 + 1 + 2 + 3 + 2)/5 = 8/5.

\item  \textbf{Problem of Time Average}: It takes only a single path. If this is one of the \textbf{rare} bad path then avg may be not reprsentitive


	
\end{itemize}	    
    
\end{frame}

\begin{frame}
\frametitle{Time Average versus Ensemble Average}
\framesubtitle{\textbf{\textit{Interpretation: Ensemble Average}}}
\begin{itemize}
\item Consider the FCFS scenario again 
\item \textbf{Time Average}: at time 1 there is some probability that the system
is still empty and there is some probability that the system contains one job.
Same for time 2,3,4....
\item Calculate the $E [N (i)] = i p(i) =$ expectation that the system have $i$ jobs
\item then apply 
$$p_i = \lim_{t \rightarrow \infty} \: P\{ N(t) = i \}$$

\end{itemize}	    
    
\end{frame}

\section{Simulation}

\begin{frame}
\frametitle{How to Simulate }
\framesubtitle{\textbf{\textit{}}}
\begin{itemize}
\item For ergodic system ( positive recurrent, aperiodic, \& irreducible) both are equal
\begin{itemize}
\item \textbf{irreducibility}: a processcan reach from any state to any other state (think of the state as the number of jobs in the system). This
is important for ensuring that the choice of initial state does not matter.

\item  \textbf{positive recurrent :} if for any state $i$, the state is revisited infinitely often, and the mean time between visits
to state $i$ (renewals) is finite. Furthermore, every time that we visit state i the system will probabilistically restart itself

\end{itemize} 
\item \textbf{Time Avg :} sampling a single process over a very long period of time and averaging those samples
\item \textbf{Ensemble Avg :} generating many independent processes and taking their Ensemble average at some far-out time $t$ 

\item {Ensemble Avg is better :}

\begin{itemize}
\item ensemble average can be obtained in parallel, by running simulations on different cores or different machines.

\item  independent data points allow us to generate
confidence intervals, which allow us to bound the deviation in our result
\end{itemize}

\end{itemize}	    
    
\end{frame}

\begin{frame}
\frametitle{Average Time in System }
\framesubtitle{\textbf{\textit{}}}
\begin{itemize}
\item two versions of the average time in system. 

Assume:

$T_i$ is the time in system of the $i$th arrival 

$ A(t)$ is the number of arrivals by time $t$

\item \textbf{For Time Avg}

$$ \overline{T}^{Time Avg.} = {{\lim}_{t \rightarrow \infty}} 
\frac{{{\sum}_{i=1}}^{A(t) T_i} }{a(t)} $$

\item \textbf{For Ensemble Avg.}
$$ \overline{T}^{Ensemble} = \lim_{i \rightarrow \infty} E[T_i] $$
\end{itemize}	    
    
\end{frame}
    
\end{document}