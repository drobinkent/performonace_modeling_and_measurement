%%%%%%%%%%%%%%%%%%%%%%%%%%%%%%%%%%%%%%%%%%%%%%%%%%%%%%%%%%%%%%%%%%%%%%%
%
%   Presentation of Beamer UNL Theme
%   Beamer Presentation by Chris Bourke
%
%%%%%%%%%%%%%%%%%%%%%%%%%%%%%%%%%%%%%%%%%%%%%%%%%%%%%%%%%%%%%%%%%%%%%%%

\documentclass{beamer}

\usetheme[hideothersubsections]{UNLTheme}
\usepackage[postscript]{ucs}
\usepackage[utf8x]{inputenc}
\usepackage{amsmath}
\usepackage{amssymb}
\usepackage{amsthm}
\usepackage{mathtools}
\title{Performance Modeling and
Design of Computer Systems- Ch 13 \\
M/M/1 and PASTA}
\author{Debobroto Das Robin} %
\institute{Kent State University}
\date{Spring 2020}




\begin{document}

%{% open a Local TeX Group
%\setbeamertemplate{sidebar}{}
\begin{frame}
        \titlepage
        \begin{center}
    \href{mailto:drobin@kent.edu}{\color{blue}{\texttt{drobin@kent.edu}}}
        \end{center}
\end{frame}

\begin{frame}
\frametitle{Overview} % Table of contents slide, comment this block out to remove it
\tableofcontents % Throughout your presentation, if you choose to use \section{} and \subsection{} commands, these will automatically be printed on this slide as an overview of your presentation
\end{frame}



\section The M/M/1 Queue }



\begin{frame} 
\frametitle{  The M/M/1 Queue }
\framesubtitle{\textbf{\textit{}}}
\begin{itemize}
\item  \textbf{Continuous-Time Markov Chain (CTMC)}:  a continuous-time
stochastic process $\{X(t), t \geq 0\}$ s.t., $ \forall s, t \geq 0$ and $ \forall i, j, x(u)$,
$$P \{X(t + s) = j | X(s) = i, X(u) = x(u), 0 \leq u \leq s\}  $$
$$= P \{X(t + s) = j | X(s) = i\} (by M.P.) $$
$$ = P \{X(t) = j | X(0) = i\} = P_{ij} (t) (stationarity)$$
\item We assume throughout that the state space is countable (though continuous)
\item  $\tau_i =$ time until the CTMC leaves state $i$, given that it is currently in state i. 
\item $\tau_i $ is memoryless and exponentially distributed

\end{itemize}

\end{frame}




\end{document}